\section{Data Cleaning Methodology}
To ensure the reliability of our analysis, we implemented a comprehensive data cleaning approach consisting of two main components: outlier removal and missing value imputation.

\subsection{Outlier Detection and Removal}
Our outlier detection strategy employed three complementary approaches:

\paragraph{Statistical Outlier Detection}
We utilized the Interquartile Range (IQR) method with variable-specific thresholds:
\begin{itemize}
    \item Standard threshold ($3 \times IQR$) for psychological measures (mood, arousal, valence)
    \item More permissive threshold ($5 \times IQR$) for app usage times to accommodate natural variability
\end{itemize}

\paragraph{Domain-based Validation}
We enforced known valid ranges for each variable type:
\begin{itemize}
    \item Mood: [1, 10] scale
    \item Arousal/Valence: [-2, 2] scale
    \item Activity: [0, 1] scale
    \item Screen time: [0, 7200] seconds (maximum 2 hours per event)
    \item Call/SMS: Binary values \{0, 1\}
\end{itemize}

\paragraph{Physiological Validation}
We identified and removed physiologically improbable changes in psychological measures:
\begin{itemize}
    \item Mood: Maximum change of 9 points per 5-minute interval
    \item Arousal/Valence: Maximum change of 4 points per 5-minute interval
\end{itemize}

This multi-faceted approach resulted in the removal of 13,735 outliers (approximately 5\% of the dataset), significantly improving data quality while preserving the underlying patterns.

\subsection{Missing Value Imputation}
We developed a time-aware, user-specific imputation strategy that accounts for the temporal nature of the data:

\paragraph{Short Gaps (< 6 hours)}
For brief periods of missing data:
\begin{itemize}
    \item Applied linear interpolation to maintain local trends
    \item Preserved the natural progression of psychological states
    \item Handled each user's data separately to maintain individual differences
\end{itemize}

\paragraph{Long Gaps (≥ 6 hours)}
For extended periods of missing data:
\begin{itemize}
    \item Utilized time-of-day means from the user's historical data
    \item Fell back to overall user means when historical data was unavailable
    \item Accounted for daily patterns in psychological states
\end{itemize}

\paragraph{Special Cases}
For specific variable types:
\begin{itemize}
    \item App usage and screen time: Imputed with 0 (assuming no usage)
    \item Binary variables (calls, SMS): Imputed with 0 (assuming no event)
\end{itemize}

This imputation strategy successfully handled most missing values, with only a single missing value remaining in the circumplex.valence variable. The approach preserves both temporal patterns and individual differences while maintaining the integrity of the psychological measurements.
